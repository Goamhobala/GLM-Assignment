
\documentclass[12pt,halfline,a4paper,]{ouparticle}

% Packages I think are necessary for basic Rmarkdown functionality
\usepackage{hyperref}
\usepackage{graphicx}
\usepackage{listings}
\usepackage{xcolor}
\usepackage{fancyvrb}
\usepackage{framed}

% Link coloring
\hypersetup{breaklinks=true,
            bookmarks=true,
            pdfauthor={},
            pdftitle={STA304XS - Assignment 2: Machine Learning}
            }


%% To allow better options for figure placement
%\usepackage{float}

% Packages that are supposedly required by OUP sty file
\usepackage{amssymb, amsmath, geometry, amsfonts, verbatim, endnotes, setspace}

% use upquote if available, for straight quotes in verbatim environments
\IfFileExists{upquote.sty}{\usepackage{upquote}}{}

% Macros for dealing with affiliations, footnotes, etc.
\makeatletter
\def\Newlabel#1#2#3{\expandafter\gdef\csname #1@#2\endcsname{#3}}

\def\Ref#1#2{\@ifundefined{#1@#2}{???}{\csname #1@#2\endcsname}}

\newcommand*\samethanks[1][\value{footnote}]{\footnotemark[#1]}

\newcommand*\ifcounter[1]{%
  \ifcsname c@#1\endcsname
    \expandafter\@firstoftwo
  \else
    \expandafter\@secondoftwo
  \fi
}

\newcommand*\thanksbycode[1]{%
  \ifcounter{FNCT@#1}
    {\samethanks[\value{FNCT@#1}]}
    {\thanks{\Ref{FN}{#1}}\newcounter{FNCT@#1}\setcounter{FNCT@#1}{\value{footnote}}}
}

% Create labels for Addresses if the are given in Elsevier format

% Create labels for Footnotes if the are given in Elsevier format

% Part for setting citation format package: natbib

% Part for setting citation format package: biblatex

% Pandoc syntax highlighting
\usepackage{color}
\usepackage{fancyvrb}
\newcommand{\VerbBar}{|}
\newcommand{\VERB}{\Verb[commandchars=\\\{\}]}
\DefineVerbatimEnvironment{Highlighting}{Verbatim}{commandchars=\\\{\}}
% Add ',fontsize=\small' for more characters per line
\usepackage{framed}
\definecolor{shadecolor}{RGB}{248,248,248}
\newenvironment{Shaded}{\begin{snugshade}}{\end{snugshade}}
\newcommand{\AlertTok}[1]{\textcolor[rgb]{0.94,0.16,0.16}{#1}}
\newcommand{\AnnotationTok}[1]{\textcolor[rgb]{0.56,0.35,0.01}{\textbf{\textit{#1}}}}
\newcommand{\AttributeTok}[1]{\textcolor[rgb]{0.13,0.29,0.53}{#1}}
\newcommand{\BaseNTok}[1]{\textcolor[rgb]{0.00,0.00,0.81}{#1}}
\newcommand{\BuiltInTok}[1]{#1}
\newcommand{\CharTok}[1]{\textcolor[rgb]{0.31,0.60,0.02}{#1}}
\newcommand{\CommentTok}[1]{\textcolor[rgb]{0.56,0.35,0.01}{\textit{#1}}}
\newcommand{\CommentVarTok}[1]{\textcolor[rgb]{0.56,0.35,0.01}{\textbf{\textit{#1}}}}
\newcommand{\ConstantTok}[1]{\textcolor[rgb]{0.56,0.35,0.01}{#1}}
\newcommand{\ControlFlowTok}[1]{\textcolor[rgb]{0.13,0.29,0.53}{\textbf{#1}}}
\newcommand{\DataTypeTok}[1]{\textcolor[rgb]{0.13,0.29,0.53}{#1}}
\newcommand{\DecValTok}[1]{\textcolor[rgb]{0.00,0.00,0.81}{#1}}
\newcommand{\DocumentationTok}[1]{\textcolor[rgb]{0.56,0.35,0.01}{\textbf{\textit{#1}}}}
\newcommand{\ErrorTok}[1]{\textcolor[rgb]{0.64,0.00,0.00}{\textbf{#1}}}
\newcommand{\ExtensionTok}[1]{#1}
\newcommand{\FloatTok}[1]{\textcolor[rgb]{0.00,0.00,0.81}{#1}}
\newcommand{\FunctionTok}[1]{\textcolor[rgb]{0.13,0.29,0.53}{\textbf{#1}}}
\newcommand{\ImportTok}[1]{#1}
\newcommand{\InformationTok}[1]{\textcolor[rgb]{0.56,0.35,0.01}{\textbf{\textit{#1}}}}
\newcommand{\KeywordTok}[1]{\textcolor[rgb]{0.13,0.29,0.53}{\textbf{#1}}}
\newcommand{\NormalTok}[1]{#1}
\newcommand{\OperatorTok}[1]{\textcolor[rgb]{0.81,0.36,0.00}{\textbf{#1}}}
\newcommand{\OtherTok}[1]{\textcolor[rgb]{0.56,0.35,0.01}{#1}}
\newcommand{\PreprocessorTok}[1]{\textcolor[rgb]{0.56,0.35,0.01}{\textit{#1}}}
\newcommand{\RegionMarkerTok}[1]{#1}
\newcommand{\SpecialCharTok}[1]{\textcolor[rgb]{0.81,0.36,0.00}{\textbf{#1}}}
\newcommand{\SpecialStringTok}[1]{\textcolor[rgb]{0.31,0.60,0.02}{#1}}
\newcommand{\StringTok}[1]{\textcolor[rgb]{0.31,0.60,0.02}{#1}}
\newcommand{\VariableTok}[1]{\textcolor[rgb]{0.00,0.00,0.00}{#1}}
\newcommand{\VerbatimStringTok}[1]{\textcolor[rgb]{0.31,0.60,0.02}{#1}}
\newcommand{\WarningTok}[1]{\textcolor[rgb]{0.56,0.35,0.01}{\textbf{\textit{#1}}}}

% tightlist command for lists without linebreak
\providecommand{\tightlist}{%
  \setlength{\itemsep}{0pt}\setlength{\parskip}{0pt}}



\usepackage{booktabs}

\begin{document}

\title{STA304XS - Assignment 2: Machine Learning}

\author{%
%
% Code for old style authors field
%
% Add \and if both authors and author
%
%
% Code for new (elsevier) style author field
\name{Jing Yeh}
%
\email{\href{mailto:yhxjin001@myuct.ac.za}{yhxjin001@myuct.ac.za}}%
%
%
%
\and
\name{Saurav Sathnarayan}
%
\email{\href{mailto:sthsau001@myuct.ac.za}{sthsau001@myuct.ac.za}}%
%
%
%
%
}

\abstract{This assignment explores two machine learning problems. First,
we analyse vehicle braking performance data using decision trees and
feed-forward neural networks to predict stopping distance based on
speed, brand, and licence type. We tune model complexity and
regulariaation parameters. Second, we apply Lloyd's K-means algorithm to
segment a wine dataset into natural groupings based on their properties,
using the elbow method and silhouette scores to determine the optimal
number of clusters.}

\date{2025-10-10}

\keywords{Decision Trees, Neural Networks, K-Means Clustering, R,
Supervised Learning, Unsupervised Learning}

\maketitle



\begin{center}\includegraphics[width=400px]{leclerc} \end{center}
\newpage
\tableofcontents
\newpage

\section{Question 1: Exploratory Data
Analysis}\label{question-1-exploratory-data-analysis}

\subsection{Part (a)}\label{part-a}

\includegraphics[width=1\linewidth]{skeleton_files/figure-latex/q1a-plots-1}
\includegraphics[width=1\linewidth]{skeleton_files/figure-latex/q1a-plots-2}

\subsection{Part (b)}\label{part-b}

Based visual inspection, we could see that the faster the car was moving
(higher speed) the greater the distance required to stop the car. Speed
and distance seem to be positively and linearly correlated. We can't see
much difference between the distance required for the different types of
licence held by the drivers. However, brand wise, it looks like bRambo
outperforms their competitors.

\section{Question 2: Decision Trees \& Neural
Networks}\label{question-2-decision-trees-neural-networks}

\subsection{Part (a)}\label{part-a-1}

\includegraphics[width=1\linewidth]{skeleton_files/figure-latex/q2a-unpruned-tree-1}

\subsection{Part (b)}\label{part-b-1}

\includegraphics[width=1\linewidth]{skeleton_files/figure-latex/q2b-pruned-tree-1}
\includegraphics[width=1\linewidth]{skeleton_files/figure-latex/q2b-pruned-tree-2}

\subsection{Part (c)}\label{part-c}

Analysing the both the pruned tree plot, we can't find the Licence type
parameter being used in the decision making. We can therefore say that
the Licence type has limited effect on the braking distance needed.

\subsection{Part (d)}\label{part-d}

\begin{Shaded}
\begin{Highlighting}[]
\FunctionTok{set.seed}\NormalTok{(}\DecValTok{2025}\NormalTok{)}
\NormalTok{design\_matrix }\OtherTok{\textless{}{-}} \FunctionTok{model.matrix}\NormalTok{(Distance }\SpecialCharTok{\textasciitilde{}}\NormalTok{ .,}\AttributeTok{data=}\NormalTok{braking\_data, }\AttributeTok{xlev=}\ConstantTok{FALSE}\NormalTok{)}
\NormalTok{design\_matrix }\OtherTok{\textless{}{-}}\NormalTok{ design\_matrix[,}\DecValTok{2}\SpecialCharTok{:}\FunctionTok{ncol}\NormalTok{(design\_matrix)] }\CommentTok{\# Remove intercept}

\NormalTok{Y }\OtherTok{\textless{}{-}}\NormalTok{ braking\_data}\SpecialCharTok{$}\NormalTok{Distance}
\NormalTok{columns }\OtherTok{=} \FunctionTok{seq}\NormalTok{(}\DecValTok{1}\NormalTok{, }\FunctionTok{length}\NormalTok{(Y))}
\NormalTok{train\_mask }\OtherTok{=} \FunctionTok{sample}\NormalTok{(columns, }\FunctionTok{length}\NormalTok{(Y) }\SpecialCharTok{*} \FloatTok{0.8}\NormalTok{)}

\NormalTok{Y\_train }\OtherTok{=}\NormalTok{ Y[train\_mask]}
\NormalTok{Y\_valid }\OtherTok{=}\NormalTok{ Y[}\SpecialCharTok{{-}}\NormalTok{train\_mask]}
\NormalTok{X\_train }\OtherTok{=}\NormalTok{ design\_matrix[train\_mask,]}
\NormalTok{X\_valid }\OtherTok{=}\NormalTok{ design\_matrix[}\SpecialCharTok{{-}}\NormalTok{train\_mask,]}
\end{Highlighting}
\end{Shaded}

\subsection{Part (e)}\label{part-e}

\includegraphics[width=1\linewidth]{skeleton_files/figure-latex/q2e-nn-unregularised-plots-1}

From the gradient plot, it seems like the gradient has stopped
decreasing, and hence reached a local minimum at around 4000 iterations.
We can also see this from the objective plots. The objective values have
also plateaued at around 4000 iterations.

\subsection{Part (f)}\label{part-f}

\includegraphics[width=1\linewidth]{skeleton_files/figure-latex/q2f-nn-regularised-plot-1}

Using min(valid\_error), we found that the 9th regularisation term gives
the lowest value. This value is 0.01174

\subsection{Part (g)}\label{part-g}

After selecting a regularization level, the training and validation data
are combined before fitting the final model at that regularization level
to maximize the use of all available data for training. This approach
allows the model to learn from the full dataset, improving its
generalization and predictive performance. The validation data, which
was previously held out to tune the regularization parameter, is now
included to provide more information for the final model fitting,
leading to a more robust and accurate model.

\section{Question 3: Model
Predictions}\label{question-3-model-predictions}

\subsection{Part (a)}\label{part-a-2}

\includegraphics[width=1\linewidth]{skeleton_files/figure-latex/q3a-tree-predictions-1}
\includegraphics[width=1\linewidth]{skeleton_files/figure-latex/q3a-tree-predictions-2}

\subsection{Part (b)}\label{part-b-2}

\includegraphics[width=1\linewidth]{skeleton_files/figure-latex/q3b-nn-predictions-1}

\subsection{Part (c)}\label{part-c-1}

Based on the plots, bRambo brakes consistently outperform the
Competitor's, resulting in shorter stopping distances at all speeds. The
driver's licence type was found to have a negligible effect on
performance, while vehicle speed remains the most critical factor. The
bRambo system's advantage is consistent across all scenarios, confirming
its superior safety performance.

\section{Question 4: K-Means
Clustering}\label{question-4-k-means-clustering}

\subsection{Part (a)}\label{part-a-3}

\begin{Shaded}
\begin{Highlighting}[]
\FunctionTok{set.seed}\NormalTok{(}\DecValTok{2025}\NormalTok{) }
\NormalTok{k\_means }\OtherTok{=} \ControlFlowTok{function}\NormalTok{(X,K,iterations) \{}
\NormalTok{   N     }\OtherTok{=} \FunctionTok{dim}\NormalTok{(X)[}\DecValTok{1}\NormalTok{]}
\NormalTok{   d     }\OtherTok{=} \FunctionTok{dim}\NormalTok{(X)[}\DecValTok{2}\NormalTok{]}
\NormalTok{   initial\_index }\OtherTok{=} \FunctionTok{sample}\NormalTok{(}\DecValTok{1}\SpecialCharTok{:}\NormalTok{N,K,}\AttributeTok{replace =} \ConstantTok{FALSE}\NormalTok{) }
\NormalTok{   Mus   }\OtherTok{=}\NormalTok{  X[initial\_index,,drop}\OtherTok{=}\ConstantTok{FALSE}\NormalTok{] }
\NormalTok{   ones  }\OtherTok{=} \FunctionTok{matrix}\NormalTok{(}\DecValTok{1}\NormalTok{,N,}\DecValTok{1}\NormalTok{) }
\NormalTok{   dists }\OtherTok{=} \FunctionTok{matrix}\NormalTok{(}\DecValTok{0}\NormalTok{,N,K) }

   \ControlFlowTok{for}\NormalTok{(i }\ControlFlowTok{in} \DecValTok{1}\SpecialCharTok{:}\NormalTok{iterations) \{}
\NormalTok{      dists }\OtherTok{=}\NormalTok{ dists}\SpecialCharTok{*}\DecValTok{0}
      \ControlFlowTok{for}\NormalTok{(k }\ControlFlowTok{in} \DecValTok{1}\SpecialCharTok{:}\NormalTok{K) \{}
\NormalTok{           dists[,k] }\OtherTok{=} \FunctionTok{rowSums}\NormalTok{((X}\SpecialCharTok{{-}}\NormalTok{ones}\SpecialCharTok{\%*\%}\NormalTok{Mus[k,])}\SpecialCharTok{\^{}}\DecValTok{2}\NormalTok{) }
\NormalTok{      \}}
\NormalTok{      assigned\_labels }\OtherTok{=} \FunctionTok{apply}\NormalTok{(dists,}\DecValTok{1}\NormalTok{,which.min) }
      
      \ControlFlowTok{for}\NormalTok{(k }\ControlFlowTok{in} \DecValTok{1}\SpecialCharTok{:}\NormalTok{K) \{}
       \ControlFlowTok{if}\NormalTok{(}\FunctionTok{any}\NormalTok{(assigned\_labels }\SpecialCharTok{==}\NormalTok{ k)) \{}
\NormalTok{         wh }\OtherTok{=} \FunctionTok{which}\NormalTok{(assigned\_labels }\SpecialCharTok{==}\NormalTok{ k)}
\NormalTok{         nk }\OtherTok{=} \FunctionTok{length}\NormalTok{(wh)}
\NormalTok{         Mus[k,] }\OtherTok{=} \FunctionTok{colSums}\NormalTok{(X[wh,,}\AttributeTok{drop =}\ConstantTok{FALSE}\NormalTok{])}\SpecialCharTok{/}\NormalTok{nk}
\NormalTok{       \} }\ControlFlowTok{else}\NormalTok{ \{}
\NormalTok{         centroid\_dists }\OtherTok{=} \FunctionTok{apply}\NormalTok{(dists,}\DecValTok{1}\NormalTok{,min)}
\NormalTok{         wh\_worst }\OtherTok{=} \FunctionTok{which.max}\NormalTok{(centroid\_dists)}
\NormalTok{         Mus[k,]}\OtherTok{=}\NormalTok{ X[wh\_worst,]}
\NormalTok{       \}}
\NormalTok{      \}}
\NormalTok{   \}}
   
\NormalTok{  SSQ }\OtherTok{\textless{}{-}} \DecValTok{0}
  \ControlFlowTok{for}\NormalTok{ (i }\ControlFlowTok{in} \DecValTok{1}\SpecialCharTok{:}\FunctionTok{nrow}\NormalTok{(dists)) \{}
\NormalTok{    k }\OtherTok{\textless{}{-}}\NormalTok{ assigned\_labels[i]}
\NormalTok{    SSQ }\OtherTok{\textless{}{-}}\NormalTok{ SSQ }\SpecialCharTok{+}\NormalTok{ dists[i, k]}
\NormalTok{  \}}
 
   \FunctionTok{return}\NormalTok{(}\FunctionTok{list}\NormalTok{(}\AttributeTok{z =}\NormalTok{ assigned\_labels, }\AttributeTok{D =}\NormalTok{ dists, }\AttributeTok{K=}\NormalTok{ K, }\AttributeTok{SSQ =}\NormalTok{ SSQ))}
\NormalTok{ \}}
 
\NormalTok{res\_k\_means }\OtherTok{=} \FunctionTok{k\_means}\NormalTok{(}\FunctionTok{as.matrix}\NormalTok{(dat\_2), }\DecValTok{3}\NormalTok{, }\DecValTok{20}\NormalTok{)}
\end{Highlighting}
\end{Shaded}

\subsection{Part (b)}\label{part-b-3}

\begin{Shaded}
\begin{Highlighting}[]
\NormalTok{SSC }\OtherTok{=} \ControlFlowTok{function}\NormalTok{(res\_k\_means)\{}
  
\NormalTok{  rk\_ordered }\OtherTok{=} \FunctionTok{t}\NormalTok{(}\FunctionTok{apply}\NormalTok{(res\_k\_means}\SpecialCharTok{$}\NormalTok{D, }\DecValTok{1}\NormalTok{, sort))}
\NormalTok{  r1 }\OtherTok{=}\NormalTok{ rk\_ordered[,}\DecValTok{1}\NormalTok{]}
\NormalTok{  r2 }\OtherTok{=}\NormalTok{ rk\_ordered[,}\DecValTok{2}\NormalTok{]}
\NormalTok{  s }\OtherTok{=}\NormalTok{ (r2 }\SpecialCharTok{{-}}\NormalTok{ r1) }\SpecialCharTok{/}\NormalTok{ r2}
  
\NormalTok{  K }\OtherTok{\textless{}{-}} \FunctionTok{ncol}\NormalTok{(res\_k\_means}\SpecialCharTok{$}\NormalTok{D)}
\NormalTok{  s\_bar\_k }\OtherTok{\textless{}{-}} \FunctionTok{numeric}\NormalTok{(K)}
  \ControlFlowTok{for}\NormalTok{ (k }\ControlFlowTok{in} \DecValTok{1}\SpecialCharTok{:}\NormalTok{K) \{}
\NormalTok{    s\_bar\_k[k] }\OtherTok{\textless{}{-}} \FunctionTok{mean}\NormalTok{(s[res\_k\_means}\SpecialCharTok{$}\NormalTok{z }\SpecialCharTok{==}\NormalTok{ k])}
\NormalTok{  \}}
  
\NormalTok{  s\_bar\_overall }\OtherTok{\textless{}{-}} \FunctionTok{mean}\NormalTok{(s)}
  
  \FunctionTok{return}\NormalTok{(}\FunctionTok{list}\NormalTok{(}\AttributeTok{s =}\NormalTok{ s, }\AttributeTok{s\_bar\_k =}\NormalTok{ s\_bar\_k, }\AttributeTok{s\_bar\_overall =}\NormalTok{ s\_bar\_overall ))}
\NormalTok{\}}

\NormalTok{ssc }\OtherTok{=} \FunctionTok{SSC}\NormalTok{(res\_k\_means)}
\end{Highlighting}
\end{Shaded}

\subsection{Part (c)}\label{part-c-2}

\includegraphics[width=1\linewidth]{skeleton_files/figure-latex/q4c-find-k-1}
\includegraphics[width=1\linewidth]{skeleton_files/figure-latex/q4c-find-k-2}

Both the SSQ (elbow) and silhouette score plots suggest the optimal
number of wine groupings is K=3. The elbow plot's curve flattens after
K=3, and the silhouette score peaks at this value, indicating the most
distinct and natural groupings in the data. For the retailer, this means
that grouping the wines into 3 types is a viable strategy.

\subsection{Part (d)}\label{part-d-1}

\includegraphics[width=1\linewidth]{skeleton_files/figure-latex/q4d-separation-analysis-1}

The groupings for K=3 are well separated. This is evidenced by the high
overall average silhouette score, which peaks at K=3. Furthermore,
analysing the average silhouette score for each individual cluster shows
that all three groups have high scores, confirming that each cluster is
dense and distinct from the others. Visual inspection of a 3D plot would
further confirm the clear separation between the three colored point
clouds, which we did using plotly but could not include in the report
due to limitations with the pdf format.






\end{document}
